\tsectionandpart{Brugerunders\o{}gelse og interessentanalyse}

\overlays{1}{
\begin{slide}{Metode}
\begin{itemize}
\xitem{Unders\o{}ge mulige metoder til probleml\o{}sning}
\xitem{Valg af metode}
\xitem{Dataindsamling}
\xitem{Analyse af resultater}
\newline 
\newline
\end{itemize}

En markedsanalyse er ikke et sp\o{}rgsm\aa{}l om at f\aa{} \newline
"flest stemmer", men om at tage en rigtig beslutning, \newline
p\aa{} baggrund af den tilegnede viden.
\end{slide}
}

\overlays{1}{
\begin{slide}{Initierende problem}
	\emph{Er der behov for udvikling af et system, der har til form\aa{}l at fjernstyre elektriske apparater via en telefon?}
	\newline
	\begin{itemize}
		\xitem{Er der behov?
		\newline
		Dette unders\o{}ges ved at foretage en brugerunders\o{}gelse}
		\newline
		\xitem{Er der mulighed?
		\newline
		Dette unders\o{}ges ved at foretage en interessentanalyse, og ved at unders\o{}ge, hvilke krav, der g\o{}r sig g\ae{}ldende for et eventuelt produkt}
	\end{itemize}
\end{slide}
}

\overlays{1}{
\begin{slide}{Er der behov? -Brugerunders\o{}gelse}
\underline{Unders\o{}gelsestyper:}
\begin{itemize}
	\xitem{	Eksplorative	- udforske}
	\xitem{Deskriptive 	- beskrive}
	\xitem{Kausale		- sammenh\ae{}nge}
	\newline
\end{itemize}
Da det \o{}nskes unders\o{}gt, hvorvidt der er interesse for produktet, udf\o{}res en delvist deskriptiv, delvist kausal unders\o{}gelse.\newline
%Deskriptiv fordi den beskriver en holdning til problemet, kausal, fordi den unders\o{}ger interessen ift. forskellige andre parametre, som f.eks. pris og udnyttelsesmuligheder.

\underline{Unders\o{}gelsesmetoder:}
\begin{itemize}
	\xitem{Kvalitativ unders\o{}gelse}
	\xitem{Kvantitativ unders\o{}gelse}
	\newline
\end{itemize}
I forhold til dette projekt, er en kvantitativ unders\o{}gelse oplagt.
\end{slide}
}
\overlays{1}{
\begin{slide}{Sp\o{}rgeskema}
\begin{itemize}
\xitem Opstille hypoteser
\xitem Udforme sp\o{}rgeskema
\xitem Udf\o{}re dataindsamling
\xitem Analysere resultatet ud fra hypoteserne
\end{itemize}
\end{slide}
}

\overlays{1}{
	\begin{slide}{M\aa{}l for unders\o{}gelsen}
	\underline{Hovedm\aa{}l:} 
	\begin{itemize}
		\xitem Er der et marked for produktet?
		\newline
		\newline
	\end{itemize}
	
	\underline{Delm\aa{}l:}
	\begin{itemize}
		\xitem Hvor stor interesse har sommerhusejere i produktet?
		\xitem Hvad er forbrugerne villige til at betale for produktet?
	\end{itemize}
	\end{slide}
}

\overlays{1}{
	\begin{slide}{Er der et marked for produktet?}
	\rput[lt](1,0){\includegraphics[width=8cm]{../Brugerundersoegelse/Diagrammer/interessefordeling.eps}}
	\vspace{4.5cm}
		\begin{itemize}
		\xitem{7 \% har meget stor interesse}
		\xitem{26\% har stor interesse}
		\newline
	\end{itemize}
	Disse tal indikerer, at der er et marked for produktet.
	\end{slide}
}

\overlays{1}{
	\begin{slide}{Sommerhusejeres interesse i produktet}
	\underline{P\aa{} en skala fra 1-5 har:}
	\begin{itemize}
		\xitem 25\% sagt 3
		\xitem 75\% sagt 4
		\newline
		\newline
	\end{itemize}
	Det vil sige, at der generelt er en stor interesse i produktet.
	\end{slide}
}


\overlays{1}{
	\begin{slide}{Produktets pris}
	\rput[lt](1,0){\includegraphics[width=8cm]{../Brugerundersoegelse/Diagrammer/prisniveau.eps}}
	\vspace{4.5cm}
	\begin{itemize}
		\xitem{Af de ca 80\%, der er interesseret i produktet, vil over halvdelen give ca. 200 kr. for det}
	\end{itemize}
	\end{slide}
}

\overlays{1}{
	\begin{slide}{Fejlkilder og usikkerhed}
	\underline{Fejlkilder:}
	\begin{itemize}
	\xitem Tilf\ae{}ldig, ikke repr\ae{}sentativ
	\xitem Eventuelt uklare sp\o{}rgsm\aa{}l
	\xitem Tolkning
	\end{itemize}
	\bigskip
	\underline{Usikkerhed:}
	\begin{itemize}
	\xitem Validitet - systematisk afvigelse
	\xitem Realibilitet - unders\o{}gelsens p\aa{}lidelighed
	\xitem Statistisk usikkerhed - for f\aa{} adspurgte
	\end{itemize}
	\end{slide}
}

\overlays{1}{
	\begin{slide}{Er der mulighed? - Interessentanalyse}
	\begin{itemize}
		\xitem {Spekulativ analyse}
		\xitem {interview med potentielle interssenter}
		\newline
		\newline
	\end{itemize}
	I dette projekt udf\o{}res en spekulativ analyse.
	Det \o{}nskes unders\o{}gt, hvilke interessenter, der er til projektet, samt i hvilken grad, de p\aa{}virker projektet.
	\end{slide}
}

\overlays{1}{
	\begin{slide}{Analysemodeller}
	\underline{Karaktermodellen}\\ 
	Inddeler efter p\aa{}virkningens karakter:\\ 
	magt, legitimitet og/eller vigtighed\bigskip

	\underline{SMC-modellen}\\ 
	Inddeler efter type:\\ 
	stat, marked og civilsamfund\bigskip

	\underline{P\aa{}virkningsmodellen}\\
	Inddeler efter deres m\aa{}de at p\aa{}virke p\aa{}:\\ variable/p\aa{}virkelige og konstante/up\aa{}virkelige
	\end{slide}
}


\overlays{1}{
	\begin{slide}{Resultat af interessentanalyse}
	\begin{itemize}
	\xitem Klarhed over, hvilke regelgivende interessenter, der skal tages hensyn til
	\xitem Forst\aa{}else af forbrugernes p\aa{}virkning p\aa{} projektet
	\end{itemize}

	%\rput[lt](1,0){\includegraphics[height=5cm]{../Interessentanalyse/Nymetode/interessent_skema.eps}}
	\end{slide}
}


\overlays{1}{
	\begin{slide}{Opsamling p\aa{} problemanalyse}
	Det er blevet bekr\ae{}ftet, at der er behov for udvikling af systemet.\bigskip
	\begin{itemize}
		\xitem {Der er ikke stor konkurrence p\aa{} markedet}
		\xitem {Der er interesse for produktet blandt forbrugerne}
		\xitem {Der er nu klarhed over interessenternes p\aa{}virkning}
	\end{itemize}
	\end{slide}
}